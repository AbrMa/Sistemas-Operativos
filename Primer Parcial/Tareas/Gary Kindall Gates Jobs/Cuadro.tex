\documentclass[a4paper,12pt]{article}
\usepackage[spanish]{babel}
\usepackage[utf8]{inputenc}
\usepackage{booktabs}
\usepackage{dirtytalk}
\usepackage{graphicx}
\usepackage{dirtytalk}
\usepackage[T1]{fontenc}
\usepackage[document]{ragged2e}

\begin{document}

\title{\Large Instituto Politécnico Nacional\\Escuela Superior de Cómputo\\Sistemas Operativos\\ Gary Kildall, Bill Gates y Steve Jobs  \\Alumno: Meza Zamora Abraham Manuel\\Profesor: Ephra\'in Herrera Salgado}
\date{}
\maketitle

\section{Cuadro comparativo} 

\begin{center}
\begin{tabular}{ |c|c|c|c| } 
 \hline
\textbf{Personaje }& \textbf{Acierto} & \textbf{Error} &\textbf{Lo que yo hubiera hecho} \\ 
 \hline
Kindall & Creación de CP/M  & Ignorar a IBM & Tener en cuenta  \\
& y Bios &  & la estrategia de negocios \\ 
 \hline
Gates & Capitalizar el & Sus prácticas  & Sinceramente nada \\
 & error de Kindall & Monopólicas &  \\
 \hline
Jobs & Implementar GUI & Hablar del GUI  & Tener un acuerdo \\
& &con Gates & de confidencialidad \\
 \hline
\end{tabular}
\end{center}

\section{Personaje} 
\begin{itemize}
\item \textit{Kindall} técnicamente era muy bueno, ya que ideo el sistema de BIOS, que se sigue usando actualmente en todos los ordenadores de la actualidad, así también el desarollo de su sitema operativo, nos da a conocer su elevado nivel técnico. Sin embargo su mayor error fue descuidar sus estrategias de negocio. Yo no lo admiro por sus capacidades estratégicas, ya que con el desarrollo de sus sitema operativo pudo haber cerrado un muy buen trato con IBM, pero no lo hizo.
\item En mi opinión \textit{Gates} fue el mejor de estos tres en los dos aspectos. Era ya desde una temprana edad una persona muy inteligente, y abandonó Harvard para crear su compañía. Ahí tomó las mejores decisiones en cuanto a negocios. La principal fue aprovecharse del error de Kindall. Pero no terminó ahí, si no que además, capitalizó mucho su sistema operativo, al cerrar muchos tratos con fabricantes y posicionar a windows, hasta la actualidad, como el sistema operativo más usado.
\item Técnicamente a \textit{Steve Jobs} no se le puede admirar mucho, ya que su función principal era de creativo y visionario. Creo que estos puntos anteriores son los más importantes, ya que la generación de ideas disruptivas, acompañada de una estrategia de publicidad bastante agresiva, lograron en ese entonces posicionar a Apple como una empresa de computadoras distinta. Si bien es debatible el origen de esas ideas como mouse, GUI, y actualmente tablets. A ellos es a quien se les puede atribuir la popularización de estos elementos. Acercando al usuario normal a las computadoras, haciendo que la interacción sea más natural.

\end{itemize}
\section{PL/M} 
El Programming Language for Microcomputers es un lenguaje de alto nivel desarrollado por Kindall, en 1973. Fue creado a raíz de la solicitud de Hank Smith que trabajaba en Intel, para los microprocesadores de esta marca.

A diferencia de otros lenguajes de su misma época no disponía de rutinas estándar de entrada o salida. A cambio incluía características que conectaban con el hardware a nivel bajo específico de su microprocesador objetivo, como acceso directo a cualquier ubicación en memoria, a los puertos de E/S o a las banderas de interrupción del procesador de manera muy eficaz. PL/M fue el primer lenguaje de programación de alto nivel para microprocesadores, y el lenguaje en el que se desarrolló originalmente el sistema operativo CP/M.


\end{document}